\documentclass{article}
\usepackage{blindtext}
\usepackage{multicol}
\usepackage{amsmath}

\usepackage{babel}
\usepackage[T1]{fontenc}

\usepackage{geometry}
 \geometry{
 a4paper,
 total={170mm,257mm},
 left=20mm,
 top=20mm,
 }

\renewcommand{\refname}{Referencias}
\renewcommand{\tablename}{Tabla}

\title{Informe}
\author{Cristian Battcock}
\date{2023-06-19}

\usepackage{Sweave}
\begin{document}
\maketitle


\Sconcordance{concordance:Informe.tex:Informe.Rnw:%
1 23 1 1 0 17 1 2 2 4 0 1 2 15 1}



\begin{itemize}
\item Instalar Knitr si fuese necesario
\item Instalar una distribucion de LaTex: tinytex
\item Instalar paquetes de LaTex mediante tinytex
\end{itemize}

<<>>
if(!require(tinytex)){
    install.packages('tinytex')
    library(tinytex)
}

\begin{Schunk}
\begin{Sinput}
> tinytex::tlmgr_install('blindtext')
\end{Sinput}
\end{Schunk}


\section{Introduccion}
En el presente informe se expondran los resultados surgidos del analisis de los datos dispuestos en un repositorio web, publicada por el Ministerio de Seguridad de la Nación de la República Argentina, a través de la Dirección Nacional de Estadística Criminal, que pone a disposición los registros administrativos del Sistema de Alerta Temprana de Suicidios (SAT-SS) del Sistema Nacional de Información Criminal, comprendida entre los años 2017/2021.


\section{Objetivos}
El presente informe es el resultado de los objetivos planteados en el proyecto que lo origino:
\begin{enumerate}
\item Exponer la cantidad de suicidios llevados a cabo en la Republica Agentina en el periodo comprendido entre los años 2017 al 2021, diferenciado por sexo.
\item Exhibir la distribucios de los casos de suicidios por franjas etarias.
\item Mostrar la distribucion de los suicidios por jurisdicciones provinciales.
\end {enumerate}


\end{document}
